\chapter{پیشنهادات و جمع‌بندی}

\section{کارهای آینده}

از مهم‌ترین اقداماتی که برای بهبود این پروژه می‌توان انجام داد، حل چالش بوجود آمده در زمینه پرس‌وجوهای مکانی مونگو است. به این‌صورت که می‌توان روشی که فصل 5 پیشنهاد شد را پیاده‌سازی کرد و تاثیر آن را بر عملکرد سامانه اندازه گرفت. زیرا همان‌طور که توضیح داده شده بود، این چالش در عمل چندان مشکل‌ساز نخواهد بود، اما درصورتی‌که اصلاح آن با روش پیشنهاد شده تاثیر مخرب ناچیزی بر عملکرد سامانه داشته باشد، اصلاح و غلبه بر آن مفید خواهد بود.\\

علاوه بر این، همانطور که تاکید شد، وظیفه اصلی این پروژه ارائه پیشنهاداتی از محصولات فروشگاه‌ها به کاربرانی است که یا از نظر موقعیت مکانی، احتمال گذر آن‌ها از آن فروشگاه‌ها بالاست و یا از پیش، علاقه‌مندی آن‌ها به آن محصولات مشخص شده است. موارد دیگری نیز می‌توان بعنوان کارهای فراتر از این پروژه انجام داد؛ برای مثال،

\begin{enumerate}

	\item می‌توان با تحلیل موقعیت مکانی هر کاربر در طول روز، مکان‌هایی که احتمالا فروشگاه هستند، ولی در این سامانه هنوز ثبت نشده‌اند را بعنوان نقطه مورد علاقه  در سامانه ثبت کرد تا پس از بررسی پشتیبان سامانه، بعنوان فروشگاه اضافه شود.
	\item پس از توافق با شرکت‌های ارائه دهنده خدمات تاکسی اینترنتی، می‌توان برای مسافرین این تاکسی‌ها درطول مسیر، پیشنهاداتی را ارائه داد که در صورت توقف برای خرید از فروشگاه پیشنهاد شده، هزینه سفر آن‌ها بصورت رایگان و از طرف این سامانه پرداخت شود؛ یا این‌که هزینه دو قسمت سفر آن‌ها (از مبدا تا فروشگاه و از فروشگاه تا مقصد اولیه) برابر با هزینه سفر مستقیم از مبدا تا مقصد تمام شود و هزینه اضافی درپی نداشته باشد.
	\item علاوه بر پیشنهاد مبتنی بر موقعیت مکانی و محصولات مشابه موردعلاقه کاربر، می‌توان پیشنهادات را از زنجیره‌هایی از مشتریانی که به فروشگاه‌های مشابهی مراجعه کرده‌اند نیز بدست آورد. در اینصورت، این سامانه عملا از ویژگی‌های یک شبکه اجتماعی مبتنی بر مکان برخوردار خواهد شد.
	
\end{enumerate}

\newpage

\section{جمع‌بندی}

امروزه بسیاری از سامانه‌های تجاری، اقدام به ارائه پیشنهادات به کاربران خود مطابق با سلیقه آن‌ها می‌کنند. اگر خدمات ارائه شده حضوری باشند و در ارائه این پیشنهادات، موقعیت مکانی کاربران بعنوان یک شاخص مهم مورد توجه قرار گیرد، می‌تواند مشتریان قابل توجهی به خود جذب نماید. سامانه تولید شده در این پروژه شامل اجزای مختلفی از جمله زیرساخت سامانه، نرم‌افزار کاربردی تلفن همراه هوشمند و پنل تحت وب برای مدیران است که باید در تعامل با یکدیگر، اهداف اصلی تعیین شده برای این پروژه را محقق نمایند. خدمت ارائه شده از طریق این نرم‌افزار، نمایش اطلاعات محصولات و فروشگاه‌های متنوع به کاربران علاقه‌مند و ارائه کوپن‌های تخفیف به آن‌هاست. درفرآیند تولید این سکوی نرم‌افزاری، چالش‌های مختلفی اعم از چالش‌های فرآیندی (از جمله نحوه دسته‌بندی وظایف)،‌ الگوریتمی (مانند نحوه محاسبه علاقه‌مندی‌ها در پایگاه‌داده استفاده شده) و معماری (مانند نحوه انتقال داده‌های محلی در طول نرم‌افزار تلفن همراه هوشمند کاربر) وجود داشته که سعی شده تا حد امکان با راه‌حل‌های مناسبی رفع شده و مسیر توسعه این سامانه هموار شود. کار انجام شده‌ی پیشِ‌رو، نقطه پایانی برای این پروژه نبوده و می‌توان ایده‌های زیادی برای پیشرفت و مسیر آینده این پروژه متصوّر شد؛ از جمله مواردی که در بخش قبلی توضیح داده شد.\\

کدهای پیاده‌سازی شده این سامانه در سه منبع از سامانه کنترل نسخه گیت‌هاب در آدرس‌های زیر موجود است:

\begin{enumerate}
	\item کد سامانه اصلی: \url{https://github.com/mmkhmmkh/LBSR_backend}
	\item کد نرم‌افزار کاربردی تلفن‌های هوشمند: \url{https://github.com/mmkhmmkh/LBSR_rn}
	\item کد پنل مدیران: \url{https://github.com/mmkhmmkh/LBSR_admin}
\end{enumerate}

