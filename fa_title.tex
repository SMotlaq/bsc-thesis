%% -!TEX root = AUTthesis.tex
% در این فایل، عنوان پایان‌نامه، مشخصات خود، متن تقدیمی‌، ستایش، سپاس‌گزاری و چکیده پایان‌نامه را به فارسی، وارد کنید.
% توجه داشته باشید که جدول حاوی مشخصات پروژه/پایان‌نامه/رساله و همچنین، مشخصات داخل آن، به طور خودکار، درج می‌شود.
%%%%%%%%%%%%%%%%%%%%%%%%%%%%%%%%%%%%
% دانشکده، آموزشکده و یا پژوهشکده  خود را وارد کنید
\faculty{دانشکده مهندسی برق}
% گرایش و گروه آموزشی خود را وارد کنید
\department{گروه کنترل}
% عنوان پایان‌نامه را وارد کنید
\fatitle{طراحی  و ساخت ساعت مچی هوشمند
\\[.75 cm]
  با قابلیت تحلیل حرکات دست و پایش سلامت}
% نام استاد(ان) راهنما را وارد کنید
\firstsupervisor{دکتر محمداعظم خسروی}
%\secondsupervisor{استاد راهنمای دوم}
% نام استاد(دان) مشاور را وارد کنید. چنانچه استاد مشاور ندارید، دستور پایین را غیرفعال کنید.
%\firstadvisor{دکتر بهادر بخشی}
%\secondadvisor{استاد مشاور دوم}
% نام نویسنده را وارد کنید
\name{سلمان }
% نام خانوادگی نویسنده را وارد کنید
\surname{عامی مطلق}
%%%%%%%%%%%%%%%%%%%%%%%%%%%%%%%%%%
\thesisdate{مرداد 1401}

% چکیده پایان‌نامه را وارد کنید
\fa-abstract{
امروزه تجهیزاتی از قبیل تلفن‌همراه و ساعت‌های هوشمند جزء لاینفکی از زندگی مردم شده و استفاده‌ی رایجی دارند. از این رو بهبود تعاملات انسان و سامانه‌های هوشمند می‌تواند برگ برنده‌ای برای این صنعت باشد. یکی از جنبه‌های این تعامل، برقراری ارتباط بین ساعت هوشمند، تلفن‌همراه و پایش حرکات فیزیکی است. در این پروژه ابتدا یک ساعت هوشمند به صورت کامل طراحی شده است. این طراحی شامل سخت‌افزار، نرم‌افزار و طراحی مکانیکی بوده که از بستر بلوتوث برای برقراری ارتباط با تلفن‌همراه استفاده می‌کند. علاوه بر این، ساعت طراحی شده دارای حسگر شتاب،‌ حسگر سلامت (پالس‌اکسی‌متر)، بازر، موتور ایجاد لرزش، شارژر باتری لیتیومی، کلیدهای لمسی و صفحه‌ی نمایش است. به کمک حسگر شتاب و پردازش سیگنال آن، متغیرهای فضایی دست اندازه‌گیری شده و به کمک پیاده‌سازی دو فیلتر کالمن، اطلاعات حسگر فیلتر می‌شوند. این تشخیص حرکت برای مواردی مثل شمارش گام و روشن شدن صفحه نمایش در صورت بالا آمدن دست استفاده شده‌اند. برای پیاده‌سازی این فیلتر از تکنیک جاگذاری مقدار نهایی ضرایب استفاده شده است که باعث کاهش چشمگیر حجم محاسبات، کاهش حافظه‌ی مورد نیاز و افزایش سرعت اجرا است. پژوهش انجام شده‌ی پیش‌رو، نقطه‌ی پایانی برای این پروژه نیست و می‌توان ایده‌های زیادی برای پیشرفت و مسیر آینده‌ی این پروژه متصور شد. هدایت یک بازوی رباتیک بر اساس حرکت دست به کمک فناوری اینترنت اشیاء مثالی از این ایده‌ها است.
}


% کلمات کلیدی پایان‌نامه را وارد کنید
\keywords{ساعت هوشمند، فیلتر کالمن، ریزپردازنده، تلفن‌همراه هوشمند، دستگاه پوشیدنی}



\AUTtitle
%%%%%%%%%%%%%%%%%%%%%%%%%%%%%%%%%%
\vspace*{7cm}
\thispagestyle{empty}
\begin{center}
\includegraphics[height=5cm,width=12cm]{besm}
\end{center}